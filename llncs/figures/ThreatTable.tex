% =================================================
% Threat Table Macros

% Const Column Width
\newcommand{\ThreatTableColWidth}{5cm}

% Const Header Row Height
\newcommand{\ThreatTableHeaderRowHeight}{0.5cm}

% Const Content Row Height
\newcommand{\ThreatTableContentRowHeight}{1cm}

% Define Header Cell Makro
\newcommand{\ThreatTableHeaderCell}[1]{
\begin{minipage}[t][\ThreatTableHeaderRowHeight][c]{\ThreatTableColWidth}
\centering
\scriptsize
\textbf{#1}
\end{minipage}
}

% Define Content Cell Makro
\newcommand{\ThreatTableContentCell}[1]{
\begin{minipage}[t][][c]{\ThreatTableColWidth}
\begin{flushleft}
\tiny
#1
\newline
\end{flushleft}
\end{minipage}
}

% Define Header Row Makro
\newcommand{\ThreatTableHeaderRow}[4]{
\ThreatTableHeaderCell{#1}
&\ThreatTableHeaderCell{#2}
&\ThreatTableHeaderCell{#3}
&\ThreatTableHeaderCell{#4}
\\ \hline
}

% Define Content Row Makro
\newcommand{\ThreatTableContentRow}[4]{
\ThreatTableContentCell{#1}
&\ThreatTableContentCell{#2}
&\ThreatTableContentCell{#3}
&\ThreatTableContentCell{#4}
\\ \hline
}

%  Define Threat Table Environment
\newenvironment{ThreatTable}
{
\begin{tabular}{|c|c|c|c|}
\hline
\ThreatTableHeaderRow
{Description}
{Conflict of Interest}
{Vulnerabilites}
{Assets (Privacy Types)}
}
{\end{tabular}}


% =================================================
% Threat Table Content

\begin{landscape}
\begin{figure}
\centering
\begin{ThreatTable}

\ThreatTableContentRow
{\textbf{(Zero Day) Exploits}
\\A criminal uses (Zero Day) Exploits to obtain access to hardware or software which stores or processes
privacy sensitive data in order to get that data.}
{Criminals want to obtain access or data for personal profit, i.e. by (re-)selling the data or  by processing it themselves. 
However, criminals may have no financial interest, they could also gain personal (ego) profit by testing proof-of-concept attacks.
\\\textbf{Criminal vs Citizen}
\\Citizens only allowed Local Authorities to use their data. They want their data to be secret to others. (Additionally, citizens are
also interested in a working system, which they payed for via taxes.)
\\\textbf{Criminal vs Local Authorities}
\\Local Authorities have capital and reputation invested in a working system. Successful attacks undermine both.
\\\textbf{Criminal vs Maintenance Staff}
\\Staff members have a professional ethos and a duty to provide working systems. Successful attacks offend the former and obstruct the latter.
\\Staff member employed by L+G or contractor. The Business operation is thretened by loss of confidential data.
}
{Unsecured Hardware- or Software-Interfaces}
{\textbf{Explicit:} 1,4,6,7}

\ThreatTableContentRow
{\textbf{Man in the Middle} 
\\A criminal intercepts communication between mobile device and server or between server and application.}
{\textit{Like Exploits}}
{Unencrypted hardware or software communication}
{\textbf{Explicit:} 1,4,6,7}

\ThreatTableContentRow
{\textbf{Corrupt Employees} 
\\ An employee abuses his database access to obtain private citizen data in order to sell it to advertisers.}
{\textbf{(Corrupt) Employee vs Citizen} 
\\ Corrupt Employees want to make personal profit by selling citizen data. 
Citizens provide data for public improvement, they don't want their data to be used for other purposes, which
may lead to negative effects for themselves.
Employees in general need easy database access to do their job. But this also means easy access to privacy
sensitive data of citizens. This diametrically opposes the interest of citizens to have such information unknown
to other individuals.}
{Full database access of Employees}
{\textbf{Explicit:} 1,4,6,7}

\ThreatTableContentRow
{\textbf{Corrupt Local Authorities}
\\A member of the Local Authority abuses his access to applications 
to obtain aggregated citizen data in order to use it for illegitimate purposes, e.g. selling it.}
{\textbf{(Corrupt) Local Authority Member vs Citizen}
\\\textit{Like Corrupt Employees}}
{Full application access of Local Authorities}
{\textbf{Explicit:} 1,4,6,7}

\ThreatTableContentRow
{\textbf{Careless Citizen}
\\A careless citizen allows others (Criminals) to have unrestricted access to his mobile device. 
Hence, he creates to possibility to install spy-ware or have the device destroyed.}
{\textbf{Criminal vs Citizen}
\\Criminals want to have access to mobile devices to obtain private data of citizens in order to 
gain personal profit - or to simply render the device useless. On the other hand, citizens have a 
natural interest in keeping personal data secret in order to prevent financial loss or because having
sensitive information accessible to others violates their privacy.}
{Insufficient access rules for mobile devices}
{\textbf{Explicit:} 1,4,6,7}

\ThreatTableContentRow
{\textbf{Intransparent Data Mining}
\\Local Authorities or System Providers use their technical knowledge, data mining capabilities and 
additional data sources to obtain/create more information about citizens.}
{\textbf{Citizen vs Local Authority or System Provider}
\\Citizens only agreed to share certain sensitive data with Local Authorities and System Providers to help society.
They are not interested in negative effects as a result of such a good willing act.
However, Local Authorities and System Providers have an interest to maximize the profit of their investments.
Local Authorities could (secretly) use mined data for security or health care issues. System Providers could
(secretly) sell mined data to illegitimate customers, e.g. the SCHUFA. This could lead to repressive behaviour of law
enforcement or negative scores.}
{Unaware Citizens}
{1,2,3,4,5,6,7}

\end{ThreatTable}

{\scriptsize (\textbf{Note for Meeting:} I think we need discrete (numbered) lists of interests for each actor in the \textit{Humans} section.)}

\caption{Threat Table}
\end{figure}
\end{landscape}
